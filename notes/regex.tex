\documentclass[main.tex]{subfiles}
\begin{document}

\chapter{Regular Expressions}
\epigraph{\texttt{[A-Z][a-z]+\textbackslash{}d\{4\}}}{Chatbot+9000}

Despite Regular expressions being something that is language independent, you will need to know how to use them in a language, and since this is taught in the Ruby section of the course, we will be learning how to use Ruby's regular expressions.
There will then be a smaller section on how to use them in OCaml, but that's more syntactical than anything really substantive. 
Regex comes from the theory related very closely to finite automata which is a later chapter so for now we will do just the basics and a surface level of the topic here. 

\section{Introduction}
For those that do not know, programs live in RAM on the machine. 
Since RAM is wiped everytime you power down your machine\footnote{for the most part. There is always the \href{https://en.wikipedia.org/wiki/Cold_boot_attack}{Cold Boot Attack}}, programs and program memory is designed to live short term. 
For long term storage, we need to use hard drives, since what is written to a hard drive is saved for a much longer period of time. 
When we write to the hard drive we typically do this by writing a file and saving it. 

One defining feature of the UNIX family is the idea that everything is treated as a file, and for the most part, this works fine since everything that is stored is stored as a file. 
We have various file types and file descriptors but ultimately whenever we need to save data for long-term storage, we save it to a file (which ultimately is just a segment of bytes in memory).
This also means that whenever we need to load something from storage, we need to open up a file and read from it.

Opening a file and getting the data from it is the easy part. 
The hard part is to try and parse and make use of the data. 
Typically from a coding perspective, you open a file, read from it and then have a string of data. 
Being able to properly and efficiently parse this string is what loading typically is. 
Examples of this is reading a save file for a videogame, loading an image from a \texttt{.png} file, loading settings from a config file. 
Regardless reading and parsing strings is important, but can be hard to do.

Many would think that if you are trying to read certain data from a string, that methods that deal with strings is the solution.
If I loaded a configuration file and wanted to know if I had to set a value to \cout{true} then I would probably want to check if the string has a subtring of \cout{true}.
This is certainly one way to solve this problem, but it soon gets very inefficient with large amounts of data. 
The solution to some of our problems here is Regular Expressions, or more commonly known as RegEx.

\section{Regular Expression Basics}
At a basic level, a \textbf{Regular Expression} is a pattern that describes a set of strings. 
At a deeper level, a regular expression defines a regular language, a language which can be created from a finite state machine. 
A finite state machine will be covered in a later chapter as well.
For now however, we can think of regex as a tool (or library) used to search (and extract!) text. 

When we wish to define a pattern to describe a set of strings, there are a few things that we must consider. 
\begin{itemize}
    \item An Alphabet - An \textbf{Alphabet} is the set of symbols or characters allowed in the string. 
If our set of strings are for English words, we would have a different alphabet than one that describes a set of mandarin strings.
\item Concatenation - Since most strings are longer than a single character, we will need some way to demonstrate the concatenation of single characters to create longer strings.
\item Alternation - Being able to say one thing or another. We could say that any non-empty set is a union of 2 other sets. So I want to say that a string in set $S$ is in either $S_1$ or $S_2$ where $\{S_1,S_2\}$ is a partition of $S$.
\item Quantification - The thing about patterns is that repeat. So if I want to have a pattern, then I need to allow for repetition. 
\item Grouping - Ultimately not something that is the basis of regex, but is helpful in giving precedence to the above.
\end{itemize}

We will talk more about all of this in a future chapter. 

\section{Regular Expression In Ruby}

Now that we have an idea of what regular expressions are, let us see how we can use them in ruby, and what type of patterns we can create. 
Ruby and many other language support the POSIX-EXT standard of regular expressions which is what we will be going over here. 
Let's start out on just making our first pattern. 
Much like \texttt{String}s denoted with single or double quotes, and arrays with the \texttt{[]} symbols, and hashes using \texttt{\{\}}, patterns are surrounded by the forward-slash: \texttt{/}. 
\begin{lstlisting}[style=MyRubyStyle]
# regex.rb
p = /pattern/
puts p.class #Regexp
\end{lstlisting}

Here is the very first example of a valid regular expression pattern. 
This pattern is the string literal \cout{pattern}.
That is, this pattern describes the set of strings that contain the substring \cout{pattern}.
Not a very fun pattern, but a pattern nonetheless. 

But Great! We have a pattern. Now how do we use it? 
There are 2 main ways that I know of, one of which I like and the other I do not. We will go over the latter because it is easier version and my reason for not liking it it petty. Suppose we have our earlier pattern. 

\begin{lstlisting}[style=MyRubyStyle]
# regex-1.rb
p = /pattern/
if p =~ "pattern" then 
  puts "Matched"
else
  puts "Not matched"
end
\end{lstlisting}
If everything goes as planned, running this file will have \texttt{"Matched"} printed. Why does this happen? 
\texttt{=~} is a method associated with Regexp objects which take in a string and returns an integer or nil depending on if
the pattern can be found \textit{anywhere} in the string. If the pattern is found, it will return the index of the first 
character of the first instance of the pattern. 

\subsubsection{Regex Syntax}
Ultimately this pattern is not fun nor interesting so let's take a look at other patterns and what they match. POSIX-EXT standard (and Ruby) regular expressions have the following pattern description/syntax.
\begin{itemize}
    \item Ranges: To talk about accepting a range of characters you can use the following: \texttt{/[a-z]/}. This particular example means accept a lowercase letter from a through z. You can of course modify this for uppercase: \texttt{/[A-Z]/}, digits: \texttt{/[0-9]/}, or smaller ranges \texttt{/[A-F]/}. Ultimately any ascii range works: \texttt{[!-\&]}. It is important to note that this will only match a single character.
    \item Concatenation: we already saw this but if \texttt{/[a-z]/} matches a single character then \texttt{/[a-z][a-z]/} will match two consecutive lowercase letters. Technically \texttt{/pattern/} matches the seven character literals 'p', 'a', 't', 't', 'e', 'r', 'n' in a row.
    \item Union: To match string that may have alternative spellings, you may use the union character: \texttt{/ste(v|ph)en/}. This may be useful for alternative spellings. Be careful because the scope of the union operator goes until a parenthesis or the beginning or end of the regex. That is, \texttt{/abc|def/} will match either "abc" or "def" whereas \texttt{/ab(c|d)ef/} will match either "abcef" or "abdef". If you have many things being unioned together, you can use the bracket syntax. \texttt{[/[aeiou'/]} will match the same strings as \texttt{/a|e|i|o|u/}. Technically \texttt{/[a-z]/} is shorthand for \texttt{/a|b|c|...|y|z/} or \texttt{/[abc..xz]/}. This also means you can combine ranges. \texttt{/[a-zA-Z0-9]/} will match any alphanumeric character. 
    \item Repetitions: A pattern is typically thought as something repeating so of course there is a repeating operation. There are three types of repetitions that are supported:
    \begin{itemize}
        \item 0 or more times: Represented as a \texttt{*}, this is placed to say that the previous pattern can occur 0 or more times. For example: \texttt{/0*[0-9]/} could match any single digit number with any number of preceding zeros. (eg. "0001", "2", "00", "0", "00000000009" would all be matched). 
        \item 1 or more times: Represented as a \texttt{+}, this is placed to say that the previous pattern can occur 1 or more times. For example: \texttt{/[0-9]+/} would match any number $\ge 0$. (eg. "0", "1", "10", "123","54234543" would all be matched).  
        \item Exact or bounded repeats: Represented as \texttt{\{x\},\{x,y\},\{x,\},\{,y\}} you can make sure than a certain of number of repeats occurs. For example \texttt{/[0-9]\{2\}/} matched only 2 digit numbers like "00,"20","99". \texttt{0-9]\{3,\}} will match any number of at least 3 digits, whereas \texttt{/[0-9]\{,3\}/} will only match numbers that have at most 3 digits. Lastly \texttt{/[0-9]\{2,3\}/} will match numbers of only 2 or 3 digits. These bounds are inclusive.
        \item 0 or 1 repeats: Represented as \texttt{?}, this will check if the pattern is there or not. For example \texttt{/-?[0-9]+/} will match positive or negative integers. 
    \end{itemize}
    It is important to note that all of these modifiers only apply to the immediately preceding pattern. That is to say \texttt{/cliff*/} and \text{/(cliff)*/} will match different things. You can use parenthesises if you wish to extend the scope of any of these operators. 
    \item Negation: If you want to say \textit{anything but} a specific character, you can do so in the brackets with the carrot symbol. For example \texttt{/[\string^b][a-z]+/} says any word at least 2 characters long that does not start with the letter 'b'. 
    \item Wild card: Represented as \texttt{.}, this will match any character. That is, despite how it looks, the pattern \texttt{/[0-9].[0-9]\{2\}/} would match on both "3.30" and "3:30". 
    \item Capture Groups: see next page.
\end{itemize}
If you wish to use any of these special characters as a literal, you can escape them in the regex. That is, to actually match a float with 2 decimal places you can do so with the pattern \texttt{/[0-9]+\textbackslash{}.[0-9]\{2\}/}. 

Additionally, using the \texttt{=~} method will search the entire string for a match. If you wish, you can make sure that it starts matching at the beginning of the string with the \texttt{\^} operator or the end of the string with the \texttt{\$} operator. Consider the following:
\begin{lstlisting}[style=MyRubyStyle]
# regular_expressions.rb
if /^where/ =~ "anywhere in the string" then
  puts "matched at the begining"
end
if /where$/ =~ "anywhere in the string" then
  puts "matched at the end"
end
if /where/ =~ "anywhere in the string" then
  puts "matched somewhere in the string"
end
\end{lstlisting}

Here the only thing printed is \texttt{Matched somewhere in the string}.
\subsubsection{Capture Groups}\label{grouping_section}

Now that you can check to see if a string matches a pattern, we can move onto the more useful part: extracting out parts of a string. Suppose that you are given phone number such as (301) 405-1000\footnote{This is the university's phone number}. If you just need the area code, I suppose you could do something like \texttt{phone\_number[1..3]}. 
But what if you have a text file of phone numbers that look like
\begin{lstlisting}[style=MyRubyStyle]
(111)-111-1111
222-2222222
3333333333
\end{lstlisting}

Then we would need to use a pattern to match what each line could be, and find a way of extracting the area code. 
Regex makes this easy by having you surround the pattern you wish to capture with parenthesis. Consider the following:
\begin{lstlisting}[style=MyRubyStyle]
# capture.rb
pattern = /([0-9]{3})-[0-9]{7}/
if pattern =~ "111-1111111" then 
    puts $1
end
\end{lstlisting}

By placing the area code in parenthesis, regex knows to store the string that matches the pattern \texttt{/[0-9]{3}/}. Now where does regex store it? For Ruby, they are stored in top level variables which can be accessed with the \texttt{\$}. So \$1 referes to the first capture group. If you had multiple capture groups like \texttt{/([0-9]{3})-([0-9]{7})/} then \$2 would refer to the next captre group and so on. I am unsure how many capture groups can exist with this method.

One thing to be careful about is the fact that parenthesis are also used for scoping of things like the \texttt{*} operator so those are also captured. Capture groups are ordered by when the opening parenthesis occurs so you can nest capture groups like so: \texttt{/(([0-9])[0-9]{2})-[0-9]{7}/}. 
\begin{lstlisting}[style=MyRubyStyle]
# capture-1.rb
pattern = /(([0-9])[0-9]{2}))-[0-9]{7}/
if pattern =~ "123-4567890" then 
    puts $1
    puts $2
end
\end{lstlisting}
This will print out \texttt{123} and then \texttt{1}. 
One other important thing to note is that whenever the \texttt{=~} method is called, then these top level variables \$1, \$2, etc, are all reset. This is the main reson as to why I do not like this method of matching patterns despute how easy it is. 

The way that I prefer to match patterns is by using the \texttt{match} method. This instead returns an array of all matched groups (if any) which means I can store the results to a variable and refer to them later and not worry about data being wiped. That's it, the only reason why I do not like the previously described method.

You can actually test and see what is accepted and captured using this fun online tool called \url{http://rubular.com}. That or you can play around with the following code segments and see what happens. 

\begin{lstlisting}[style=MyRubyStyle]
# capture-2.rb
pattern = /[A-Z][A-Z]*/
strs = ["a","A","abcD","ABDC"] 
for test_string in strs
  if pattern =~ test_string
    puts "matched"
  else
    puts "not matched"
  end
end

#what if the pattern and test strings are as follows:
pattern = /a[A-Za-z]?/
strs = ["a","abd","bad"]

pattern = /^a[A-Za-z]?$/
strs = ["a","abd","bad"]

pattern = /a*b*c*d*/ # how is this different from /[a-d]/ ?
strs = ["abcd", "bad", "cad", "aaaaaacd", "bbbdddd"]

pattern = /^(..)$/
strs = ["even","odd","four","three","five"]

# an even number of vowels
pattern = /^([^aeiou]*[aeiou][^aeiou]*[aeiou][^aeiou])*$/
\end{lstlisting}

\section{Regular Expression In OCaml}
\end{document}
