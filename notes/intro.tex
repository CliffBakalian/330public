\documentclass[main.tex]{subfiles}
\begin{document}

\chapter{Intro}
\epigraph{Hello There}{General Kenobi}

I took this course many moons ago and so now I'm making notes based on what I remember from the course and my own experience playing around with programming languages. That being said, I take a pretty holistic approach to this course. That is, I assume that you have a good understanding of the previous classes you needed to get here and we will talk about how what you learn here relates to what you have learned previously.

\section{What is a language?}

After pulling out my HESP140 notes, I can say with some confidence that a formal definition for language can be simply put as a system of communication. However, after pulling out my philosophy notes\footnote{I would recommend taking philosophy of language with Alexander Williams }, I want to say that language is anything that transfers what goes on 'in our mind' to 'out of our mind.' I'm sure more important people would disagree, but eh.

What I am trying to get at is the fact that we all have thoughts and feelings, and ultimately no one knows what goes on in our heads until we express or share what we are thinking and feeling \footnote{Thomas Nagel has a fun little paper called "What is it like to be a bat" that says no matter how much we know about bats and no matter how hard we imagine what echolocation would be like, we could not experience how a bat navigates.} To be put even more succinctly: a language is a medium used to express ourselves. And in my experience, programming languages are no different. However it's not that simple. 

\section{How do we use language?}

So now that we answered what a language is, \textbf{we can now focus on one part of what this course wants to make sure you understand: how to express ourselves with a language}. To answer this question, we need to learn some new words: \textbf{semantics} and \textbf{syntax}. \textbf{semantics refers to the meaning of sentences/languages} while  \textbf{Syntax refers to the structure of the language}\footnote{Some would say this sounds like grammar. This is partially true since syntax is a subset of grammar}. Consider the following:
\begin{center}
\begin{itemize}
    \item The snow is white\\
    \item schnee ist wei{\ss}\\
    \item Precipitation comprised of ice cystrals under the temperture of $0^{\circ}$ C reflects all wavelengths of light from 400nm to 700 nm.
\end{itemize}
\end{center}
I would argue that all 3 sentences have the same meaning. That is, the semantics of the sentences are the same. Programming languages are the same. Consider the following:
\begin{verbatim}
\\java
x \% 2 == 0 ? System.out.println("Even"):System.out.println("Odd");

\* C *\
if (x \% 2 == 0){
    printf("Even\textbackslash{}n);
}else{
    printf("Odd\textbackslash{}n);
}
\end{verbatim}
The semantics of these two programs are the same, despite them looking different. This is where syntax comes into play. Since syntax deals with the rules of what is valid or not, let us take an even smaller example:
\begin{verbatim}
\\java
System.out.println("Hello, World!")

\*C*\
printf("Hello, World!"0
\end{verbatim}
Syntax deals with what is valid rules to create a sentence in a language. If we tried to write the first line in a C program, the compiler will yell at us, and had we tried to write some C code in Java, the compiler will be yell at us again. That is to express the same thing, in one language requires one thing, and in another something else. We will eventually talk about grammar, but for now just keep in mind the idea of syntax and semantics. \textbf{An array will always be an array, but the code you use to make one will differ from language to language}. That said, most languages are Turing complete (which we will also discuss later), so basically any program you make in one language, can be made in a different one which raises the following question(s).

\section{Why so many languages? Why use one over another?}

As we hopefully all know, computers only really know is machine code. But we as programmers don't really know or play around with bytecode. We use other languages which are easier to write with because they have shortcuts or macros that cover the hard and tedious stuff. Recall the assembly project in 216:
\begin{verbatim}
func: 
    push    ebp     
    mov     ebp,esp
    ; code
    mov     esp, ebp
    pop     ebp
\end{verbatim}
This particular example represents the stack frame that is added to the stack whenever a function is called. It would be terrible if we had to do this everytime we wanted to call a function, so if we can replace the aforementioned assembly with something nice:
\begin{verbatim}
    func();
\end{verbatim}

That is most languages have some way to implement or represent a function call (typically means adding parenthesises after the function name). \textbf{The idea of having special shortcuts in a language is the basis for the second point this course finds important: language features}. Different languages features is why there are so many languages and why you way want to use one language instead of another. 

\section{Language Features}

Let us consider the following Java code:
\begin{verbatim}
int sum = 0
for(int i = 0; i < 10; i++){
    sum += i
}
\end{verbatim}
Now consider the following LSIP code which does the exact same thing:
\begin{verbatim}
(defun sum (s a) (if (= a 0) s (sum (+ s a) (- a 1))))
(sum 0 10)
\end{verbatim}
Now these two code segments do the same thing, but notice that LISP's doesn't seem fun or as straight forward. We will discuss later in the course why that is, and other fun thinga about this, but for now \textbf{you just need to know that there is no such thing as iterative structures in LISP}. LISP is purely recursive (and we will learn this with OCaml) and so it has a different way to express what to do. Throughout this course, you will see this idea over and over: \textbf{some languages have certain ways to expressing things that others do not}. When talking about these features, I will try to highlight why the feature is useful and more importantly how this feature is implemented in the backend, and how you could incorporate it in other languages. \textbf{By the end of this class you should be able to make your own or at least modify programming languages to have features from other languages}. An example of this with languages and ideas you already should know would be knowing how to implement object oriented inheritance in C (hint: void pointers and structs).

\section{Conclusion}
This course focuses on a few things
\begin{center}
    \begin{itemize}
    \item How to express and learn the syntax for a programming language
    \item Learn how to use some features that languages have and how they work
    \item Knowing how language features work, being able to implement them in any language
    \end{itemize}
\end{center}
\end{document}